% !TeX root=main.tex
% در این فایل، عنوان پایان‌نامه، مشخصات خود، متن تقدیمی‌، ستایش، سپاس‌گزاری و چکیده پایان‌نامه را به فارسی، وارد کنید.
% توجه داشته باشید که جدول حاوی مشخصات پروژه/پایان‌نامه/رساله و همچنین، مشخصات داخل آن، به طور خودکار، درج می‌شود.
%%%%%%%%%%%%%%%%%%%%%%%%%%%%%%%%%%%%
% دانشگاه خود را وارد کنید
\university{علم و صنعت ایران}
% دانشکده، آموزشکده و یا پژوهشکده  خود را وارد کنید
\faculty{دانشکده مهندسی کامپیوتر}
% گروه آموزشی خود را وارد کنید
\department{گروه مهندسی نرم‌افزار}
% گروه آموزشی خود را وارد کنید
\subject{مهندسی کامپیوتر}
% گرایش خود را وارد کنید
\field{مهندسی نرم‌افزار}
% عنوان پایان‌نامه را وارد کنید
\title{ارائه سازوکاری در سکوی کوبرنیتز برای پایش و کنترل دستگاه‌های اینترنت اشیاء}
% نام استاد(ان) راهنما را وارد کنید
\firstsupervisor{دکتر محسن شریفی}
% \secondsupervisor{استاد راهنمای دوم}
% نام استاد(دان) مشاور را وارد کنید. چنانچه استاد مشاور ندارید، دستور پایین را غیرفعال کنید.
% \firstadvisor{استاد مشاور اول}
%\secondadvisor{استاد مشاور دوم}
% نام دانشجو را وارد کنید
\name{سینا}
% نام خانوادگی دانشجو را وارد کنید
\surname{شعبانی کومله}
% شماره دانشجویی دانشجو را وارد کنید
\studentID{97521351}
% تاریخ پایان‌نامه را وارد کنید
\thesisdate{تابستان ۱۴۰۲}
% به صورت پیش‌فرض برای پایان‌نامه‌های کارشناسی تا دکترا به ترتیب از عبارات «پروژه»، «پایان‌نامه» و »رساله» استفاده می‌شود؛ اگر  نمی‌پسندید هر عنوانی را که مایلید در دستور زیر قرار داده و آنرا از حالت توضیح خارج کنید.
%\projectLabel{پایان‌نامه}

% به صورت پیش‌فرض برای عناوین مقاطع تحصیلی کارشناسی تا دکترا به ترتیب از عبارات «کارشناسی»، «کارشناسی ارشد» و »دکترا» استفاده می‌شود؛ اگر  نمی‌پسندید هر عنوانی را که مایلید در دستور زیر قرار داده و آنرا از حالت توضیح خارج کنید.
\degree{کارشناسی}

\firstPage
% \besmPage
\davaranPage

%\vspace{.5cm}
% در این قسمت اسامی اساتید راهنما، مشاور و داور باید به صورت دستی وارد شوند
%\renewcommand{\arraystretch}{1.2}
\begin{center}
    \begin{tabular}{| p{8mm} | p{18mm} | p{.17\textwidth} |p{14mm}|p{.2\textwidth}|c|}
        \hline
        ردیف & سمت          & نام و نام خانوادگی           & مرتبه \newline دانشگاهی & دانشگاه یا مؤسسه                  & امضـــــــــــــا \\
        \hline
        ۱    & استاد راهنما & دکتر \newline محسن شریفی & استاد تمام                 & دانشگاه \newline علم و صنعت ایران &                   \\
        \hline
        ۲    & داور نهایی   & دکتر \newline  -        & دانشیار                 & دانشگاه \newline علم و صنعت ایران &                   \\
        \hline
    \end{tabular}
\end{center}

\esalatPage
\mojavezPage


% چنانچه مایل به چاپ صفحات «تقدیم»، «نیایش» و «سپاس‌گزاری» در خروجی نیستید، خط‌های زیر را با گذاشتن ٪  در ابتدای آنها غیرفعال کنید.
% پایان‌نامه خود را تقدیم کنید!

\newpage
\thispagestyle{empty}
% {\Large تقدیم به:}\\
% \begin{flushleft}
% {\huge
% همسر و فرزندانم\\
% \vspace{7mm}
% و\\
% \vspace{7mm}
% پدر و مادرم
% }
% \end{flushleft}


% سپاس‌گزاری
% \begin{acknowledgementpage}
% سپاس خداوندگار حکیم را که با لطف بی‌کران خود، آدمی را زیور عقل آراست.


% در آغاز وظیفه‌  خود  می‌دانم از زحمات بی‌دریغ استاد  راهنمای خود،  جناب آقای دکتر ...، صمیمانه تشکر و  قدردانی کنم  که قطعاً بدون راهنمایی‌های ارزنده‌  ایشان، این مجموعه  به انجام  نمی‌رسید.

% از جناب  آقای  دکتر ...   که زحمت  مطالعه و مشاوره‌  این رساله را تقبل  فرمودند و در آماده سازی  این رساله، به نحو احسن اینجانب را مورد راهنمایی قرار دادند، کمال امتنان را دارم.

% همچنین لازم می‌دانم از پدید آورندگان بسته زی‌پرشین، مخصوصاً جناب آقای  وفا خلیقی، که این پایان‌نامه با استفاده از این بسته، آماده شده است و همه دوستانمان در گروه پارسی‌لاتک کمال قدردانی را داشته باشم.

%  در پایان، بوسه می‌زنم بر دستان خداوندگاران مهر و مهربانی، پدر و مادر عزیزم و بعد از خدا، ستایش می‌کنم وجود مقدس‌شان را و تشکر می‌کنم از خانواده عزیزم به پاس عاطفه سرشار و گرمای امیدبخش وجودشان، که بهترین پشتیبان من بودند.
% % با استفاده از دستور زیر، امضای شما، به طور خودکار، درج می‌شود.
% \signature 
% \end{acknowledgementpage}
%%%%%%%%%%%%%%%%%%%%%%%%%%%%%%%%%%%%
% کلمات کلیدی پایان‌نامه را وارد کنید
\keywords{اینترنت اشیاء، کوبرنیتز، کوبلت مجازی، نظارت یکپارچه، پایش مقیاس پذیز}
%چکیده پایان‌نامه را وارد کنید، برای ایجاد پاراگراف جدید از \\ استفاده کنید. اگر خط خالی دشته باشید، خطا خواهید گرفت.
\fa-abstract{
    در حال حاضر، مدیریت و نظارت بر دستگاه‌های
    اینترنت اشیاء
    به یک چالش عمده تبدیل شده است. راه حل‌های موجود
    برای کنترل و نظارت بر این دستگاه‌ها اغلب ناسازگاری‌ها و محدودیت‌هایی دارند که موجب کاهش کارایی و پیچیدگی مدیریت در مقیاس بالا می‌شوند.
    به منظور حل این مسئله، این پروژه تلاش می‌کند تا با استفاده از 
    کوبرنیتز
    و پروژه
    کوبلت مجازی
    یک سازوکار جامع برای نظارت و مدیریت دستگاه‌های اینترنت اشیاء ارائه دهد. انگیزه اصلی پروژه متمرکز
    کردن کنترل و نظارت بر دستگاه‌های اینترنت اشیاء به صورت یکپارچه و موثر است. راه حل‌های کنونی اغلب ناسازگاری‌هایی
    با استانداردها و فناوری‌های مختلف دستگاه‌های اینترنت اشیاء دارند و به تنهایی قادر به ارائه یک محیط یکپارچه برای
    مدیریت و نظارت نیستند. این پروژه شامل سه بخش اصلی، یعنی
    تامین‌کننده، کنترل‌کننده و دستگاه‌ها می‌باشد.
    این بخشها با یکدیگر ارتباط برقرار میکنند تا اطلاعات مفیدی درباره دستگاههای اینترنت اشیاء مورد کنترل ارائه دهند
    و این اطلاعات را در دسترس خوشه کوبرنیتز قرار دهند.
}

\abstractPage

\newpage\clearpage