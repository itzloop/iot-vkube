% !TeX root=main.tex

\chapter{نتیجه‌گیری و کارهای آینده} \label{ch:concl}
\thispagestyle{empty}


\section{نتیجه‌گیری}
\paragraph{}
{
      در ابتدا دیدیم که روش‌های زیادی برای پایش دستگاه‌های اینترنت اشیاء بصورت متمرکز و در مقایس بالا
      وجود ندارد. سپس با ارائه روش پیشنهادی، مشهاده شد که، با کمک کوبلت مجازی، پایش‌ دستگاه‌های اینترنت اشیاء
      در مقیاس بالا بسیار ساده خواهد بود. این روش به راحتی امکان پایش هزاران دستگاه اینترنت اشیاء در یک شهر را می‌دهد.
}

\section{دستاوردها}
\paragraph{}
{
      \begin{enumerate}
            \item پایش دستگاه‌های اینترنت اشیاء در مقیاس بالا
            \item استفاده از قراردادها و فناوری‌های شناخته شده و روش‌های متعارف و مرسوم
            \item سهولت اتصال دستگاه‌های اینترنت اشیاء به فضای ابری
            \item متمرکز شدن نرم‌افزار‌های معمول در کنار دستگاه‌های اینترنت اشیاء
      \end{enumerate}
}

\section{کارهای آینده}
\paragraph{}
{
      همانطور که در قسمت ارزیابی مشاهده شد، با وجود اعداد و ارقام معقول و مناسب، همچنان نمی‌توان ادعا کرد که 
      این مسئله حل شده است. هم‌چنین معماری‌های ارائه شده به نسبت قدیمی هستند و روش‌های بسیار نوآورانه‌تری معرفی 
      شده‌اند که می‌توانند این مسئله را با سادگی بیشتری حل کنند که در ادامه به آن‌ها می‌پردازیم.
}

\subsection{امنیت}
\label{subsec:security}
\paragraph{}
{
      در این پروژه درباره امنیت حرفی زده نشد. علت این امر هم این است که هدف پروژه امکانسنجی و پیاده‌سازی بوده است.
      امنیت بعنوان یک لایه روی معماری فعلی سوار شده. در چند بخش این مسئله را مورد بررسی قرار می‌دهیم:
      \begin{enumerate}
            \item امنیت ارتباط تامین‌کننده با کوبرنیتز
            \item امنیت ارتباط راط گرافیکی با تامین‌کننده
            \item امنیت ارتباط با دستگاه‌های اینترنت اشیاء
      \end{enumerate}

      در خصوص مورد اول، این مسئله حل شده است. بستر کوبرنیتز روش‌های مناسبی برای رمز‌نگاری داده‌ها و همچنین احراز هویت دارد و فقط باید در تامین‌کننده از آن‌ها استفاده نمود.
      \\
      مورد دوم و سوم اما جای کار دارد. باید سیستم احراز هویت در تامین کننده پیاده شده و از قراردادهای امن نظیر قرارداد امنیت لایه انتقال\footnote{\lr{Transport Layer Securty Protocol (TLS)}} استفاده شود.
}

\subsection{استفاده از فناوری‌های بروز تر}
\label{subsec:newer_tech}
\paragraph{}
{
      همانطور که در بخش
      \ref{sec:system_arch} به شرح معماری پرداخته شد سامانه پرداخته شد، 
      دیدیم که این معماری از قرارداد انتقال فرا متن استفاده کرده و جنس داده‌های ارسالی هم JSON می‌باشد. 
      مزیت استفاده از این فناوری، سادگی‌ آن است. اما قرارداد‌های بروزتری وجود دارند که کارایی سامانه را بالاتر می‌برند.
      برای مثال استفاده از \lr{gRPC} بجای قرارداد انتقال فرا متن. یا حتی استفاده از قرارداد \lr{websocket}.
}

\subsection{چارچوب}
\label{subsec:framework}
\paragraph{}
{
      همانطور که در بخش
      \ref{subsec:simulator} یک شبیه‌ساز ارائه شده،
      دیدیم که یک شبیه‌ برای نمایش نحوه کارکرد معماری ارائه شده.
      این به معنی است که برای استفاده از این روش، برنامه‌نویسان باید رابط‌های کاربردی قابل برنامه‌نویسی خود را
      پیاده‌سازی کرده و با معرفی آن از طریق کوبرنیتز به تامین کننده، امکان استفاده از این روش را داشته باشند.
      برای سهولت این امر، می‌توان چارچوبی\footnote{\lr{Framework}} ارائه کرد.
      در این چارچوب، بخش‌هایی وجود خواهد داشت که به برنامه‌نویسان امکان می‌دهد کد‌های خود در آن قرار داده و مسئولیت
      اجرا و برقراری ارتباط و امنیت را به چارچوب بسپارند.
}