% !TeX root=main.tex

\chapter{نتیجه‌گیری و کارهای آینده} \label{ch:concl}
\thispagestyle{empty}


\section{نتیجه‌گیری}
\paragraph{}
{
      % در این پروژه ابتدا روش‌های مختلف حل پرسش و پاسخ تصویری مورد تحلیل واقع شد.
      % سپس روشی نو برای حل ان ارائه شد که به درک انسان از سوالات و پاسخ دادن به آن‌ها نزدیک‌‌تر است. تولید پاسخ‌ها 
      % بدون هیچ دانش خارجی و به صورت جمله در پاسخ به سوالات مربوط به یک تصویر نسبت به پاسخ به صورت تک‌کلمه
      % روشی معممول‌تر و رایج‌تر است و حل این مسئله به این روش می‌تواند سیستم را به روش‌های انسانی نزدیک‌تر کند.

      % در هسته
      % در هسته‌ی اصلی سیستم پیشنهادی مدلی با معماری
      % کدگذار-کدگشا قرار دارد.
      % قسمت کدگذار که عمده وظیفه تحلیل داده 
      % ورودی را دارد از مدل‌های از پیش آموزش‌ دیده در مسائل زبان-تصویر است که 
      % در این پژوهش از دو مدل 
      % LXMERT
      % و
      % VisualBERT
      % استفاده شده است. 
      % مدل کدگشا نیز معماری‌های متفاوتی در نظر گرفته‌شد که مدل‌های تبدیل شونده 
      % بهترین عملکرد را داشتند و به دقت مناسبی بر مجموعه داده رسیدند. 

      % برای ارزیابی هر معماری، چندین معیار در نظر گرفته شد که از 
      % ابعاد معنایی و نحوی مقادیر خروجی را ارزیابی می‌کنند.
      % پس از ارزیابی برترین معماری‌ها، این نتیجه حاصل شد که 
      % استفاده از این معماری‌ها در مجموعه داده مدنظر مفید واقع شده
      % و توانسته‌اند مسئله را با دقت خوبی حل کنند. سپس با روش های
      % ارزیابی انسانی ثابت شد که این روش به رفع ابهامات در پرسش
      % کمک بسیاری می‌کند. 

      % در نظر گرفتن این نکته حائز اهمیت است که این دقت بالا 
      % در معیار‌ها واقعی نیست زیرا مجموعه داده از داده‌های موجود در
      % دنیای بیرون بسیار متفاوت‌تر است و به صورت خودکار 
      % تولید شده‌اند و حالت طبیعی خود را ندارند. لیکن این به معنای حل
      % کامل مسئله نیست و لازم به اعمال بسیاری 
      % برای ادعای حل کامل مسئله پرسش‌وپاسخ تصویری به صورت تولید متن است. 
}

\section{دستاوردها}
\paragraph{}
{
      % در تمامی قسمت‌های این پژوهش اعم از قسمت‌های طراحی، پیاده‌سازی و ارزیابی نوآوری‌هایی مطرح شد که به شرح زیر است:
      % \begin{enumerate}
      %       \item {
      %                   ارائه روشی نو برای حل مسئله پرسش و پاسخ تصویری که در صورت حل، روشی کامل‌تر از روش‌های فعلی است. 
      %       }
      %       \item {
      %                   معرفی و استفاده از معماری‌های نو متناسب با روش پیشنهادی
      %             }
      %       \item{
      %                   اثبات درستی و فواید استفاده از این روش برای حل مسئله
      %       }
      % \end{enumerate}
}

\section{کارهای آینده}
\paragraph{}
{
      % همانطور که در قسمت ارزیابی مشاهده شد، با وجود اعداد و ارقام به نسبت بالا، همچنان نمی‌توان ادعا کرد که 
      % این مسئله حل شده است. هم‌چنین معماری‌های اراسه شده به نسبت قدیمی هستند و روش‌های بسیار نوآورانه‌تری معرفی 
      % شده‌اند که می‌توانند این مسئله را با سادگی بیشتری حل کنند. 
      % چندی از این ایده‌ها که بهبود این مسئله کمک بسیاری می‌کنند در زیر آورده شده است.
      % \begin{enumerate}
      %       \item {
      %             ارائه مجموعه داده‌ای جامع و کامل، به طوری که 
      %             پاسخ‌ها به صورت جمله و طبیعی باشند. اولین قدم برای
      %             بهبود این سیستم ارائه مجموعه داده‌ی کامل‌تری 
      %             است که بتوان معماری‌های پیچیده‌تر را بر آن آموزش داد 
      %             و همچنین به نتایج بدست آمده و نزدیک بودن آن به دنیای واقعی اطمینان بیشتری داشت. 
      %       }
      %       \item {
      %             استفاده از کدگشا‌های از پیش‌ آموزش داده‌شده، یکی دیگر از راه‌های بهبود حل، ارائه معماری‌هایی غنی‌تر 
      %             با پیچیدگی بالاتر است که قادر به حل مسئله‌ به صورت جامع‌تر باشند. از این معماری‌ها می‌توان استفاده از 
      %             مدل‌هایی نظیر 
      %             BART
      %             و یا 
      %             GPT 
      %             را نام برد. استفاده از این قبیل معماری‌ها باعث عمومی‌سازی بیشتر سیستم می‌شود. 
      %       }
      % \end{enumerate}
}