% !TeX root=main.tex

\chapter{کارهای مرتبط} \label{ch:related}
\thispagestyle{empty}


\section{مقدمه}
\label{sec:intro}
\paragraph{}
{
    در این فصل به معرفی کارهای مشابه و شرح
    رویکردهای مورد استفاده برای حل مساله‌ی 
    کنترل و پایش دستگاه‌های اینترنت اشیاء بر بستر کوبرنیتز می‌پردازیم.    
}

\section{
    روش‌های متفاوت حل مسئله کنترل و پایش دستگاه‌های اینترنت اشیاء
}
\label{sec:approaches_to_monitoring_iot}
\paragraph{}{
    در مسئله مورد بحث، کارهای زیادی انجام نشده است. در ادامه به بررسی برخی از معدود روش‌های حل این مسئله می‌پردازیم
}
\subsection{
    کنترل و پایش دستگاه‌های اینترنت اشیاء بر پایه زنجیره‌ی بلوکی
}
\label{sec:blockchain_based_monitoring}
\paragraph{}{
    در \cite{s19040856}،
    نویسندگان به بررسی مسائل مدیریت و نظارت بر دستگاه‌های اینترنت اشیاء می‌پردازند
    و از تکنولوژی زنجیره‌ی بلوکی\footnote{\lr{Blockchain}} برای حل این مسائل استفاده می‌کنند. 
    در این مقاله، نویسندگان نکاتی را در مورد مزایا و چالش‌های استفاده از زنجیره‌ی بلوکی در مدیریت 
    و نظارت بر دستگاه‌های اینترنت اشیاء بررسی می‌کنند. آنها به بررسی معماری زنجیره‌ی بلوکی و
    نحوه استفاده آن در این زمینه می‌پردازند. همچنین، روش‌های مختلف برای اعتبارسنجی و امنیت 
    دستگاه‌های اینترنت اشیاء با استفاده از زنجیره‌ی بلوکی را مورد بررسی قرار می‌دهند. در ادامه، 
    نویسندگان به بررسی موارد کاربردی مدیریت و نظارت بر دستگاه‌های اینترنت اشیاء با استفاده
    از زنجیره‌ی بلوکی می‌پردازند. آنها به بررسی روش‌هایی برای ثبت و ذخیره داده‌های مربوط به 
    دستگاه‌های اینترنت اشیاء در زنجیره‌ی بلوکی می‌پردازند و نحوه استفاده از قراردادهای هوشمند
    برای اجرای منطق کسب و کار را بررسی می‌کنند.
    \begin{figure}[H]
        \center{\includegraphics[width=\textwidth]{figs/ref_figs/blockchain_monitoring.png}}
        \caption{پایش مبتنی بر زنجیره‌ی بلوکی \cite{s19040856}}
        \label{fig:blockchain_monitoring}
    \end{figure}
}

\subsection{
    گسترش کوبرنیتز بر روی دستگا‌های لبه
}
\label{subsec:extending_kubernetes_to_edge}
\paragraph{}{
    در \cite{exteing-kubernetes-to-low-resource-edge-devices}،
    نویسندگان به بررسی نیازمندی‌های موجود برای ارتقاء خوشه‌های کوبرنیتز به دستگاه‌های لبه
    با منابع کم می‌پردازند و نشان می‌دهند که استفاده از کوبلت مجازی می‌تواند یک راه حل
    مناسب برای این امر باشد. آنها به توضیح عملکرد کوبلت مجازی، که نماینده‌های مجازی در
    خوشه کوبرنیتز است، می‌پردازند و نحوه تعامل آن با دستگاه‌های لبه را تشریح می‌کنند. در
    ادامه، نویسندگان به بررسی فناوری‌ها و پروتکل‌های مورد استفاده در کوبلت مجازی می‌پردازند
    و نحوه اتصال و مدیریت آن را توضیح می‌دهند. آنها به بررسی عملکرد استفاده از کوبلت مجازی 
    در دستگاه‌های لبه، می‌پردازند. در نهایت، نویسندگان به ارزیابی و ارائه نتایج آزمایش‌هایی
    که در ارتقاء خوشه‌های کوبرنیتز به دستگاه‌های لبه با استفاده از کوبلت مجازی انجام
    شده است، می‌پردازند. آنها به بررسی کارایی و کارایی سیستم، زمان پاسخ و نیازمندی‌های 
    منابع مورد نیاز برای استفاده از کوبلت مجازی در دستگاه‌های لبه می‌پردازند.
}

\subsection{
    پایش خانه هوشمند به کمک دستگاه‌های اینترنت اشیاء
}
\label{subsec:iot_based_monitoring}
\paragraph{}{
    در \cite{iot_based_monitoring}،
    نویسندگان به بررسی مزایا و چالش‌های استفاده از اینترنت اشیاء در خانه‌های هوشمند می‌پردازند.
    آنها به تشریح معماری سیستم نظارت و کنترل مبتنی بر اینترنت اشیاء برای خودکارسازی خانه
    می‌پردازند و نحوه ارتباط و تعامل بین اجزای مختلف این سیستم را بررسی می‌کنند. در ادامه، 
    نویسندگان به توضیح کارکردها و قابلیت‌های سیستم نظارت و کنترل مبتنی بر اینترنت اشیاء برای
    خودکارسازی خانه می‌پردازند. آنها به بررسی امکانات نظارت بر دستگاه‌های خانگی، کنترل سیستم‌های 
    روشنایی و الکترونیکی، کنترل دما و رطوبت، و همچنین مدیریت از راه دور این سیستم می‌پردازند. 
    در نهایت، نویسندگان به ارائه نتایج واقعی سیستم نظارت و کنترل مبتنی بر اینترنت اشیاء برای 
    خودکارسازی خانه پرداخته و مورد استفاده قرار گرفتن این سیستم را بررسی می‌کنند. آنها به تأثیر
    این سیستم بر کاهش مصرف انرژی، راحتی کاربران و بهبود کیفیت زندگی در خانه اشاره می‌کنند.
}

\subsection{
    پایش دستگاه‌های اینترنت اشیاء بر پایه کوبرنیتز
}
\label{subsec:iot_device_management}
\paragraph{}{‍
    در این مقاله \cite{Mlynka2022thesis}، چالش‌های فناوریکی در توسعه معماری اینترنت اشیاء بررسی می‌شود و راه‌حلی خاص بر بستر کوبرنیتز برای مدیریت میلیون‌ها دستگاه در محیط‌های مختلف پیشنهاد می‌شود. این پایان‌نامه بر تأسیس یک زیرساخت اینترنت اشیاء کامل با استفاده از ابزار کوبرنیتز تمرکز دارد که به عنوان یکی از بهترین راه‌حل‌ها برای کنترل منابع مختلف شناخته شده است و در محیط‌های ابری به‌طور گسترده برای مدیریت بارکاری نرم‌افزاری استفاده می‌شود. هدف اصلی این کار، ایجاد یک معماری اینترنت اشیاء انعطاف‌پذیر، مقیاس‌پذیر و گسترش‌پذیر با تأکید بر امنیت در تمام لایه‌های آن است. مقاله نیز به بررسی فناوری‌های موردنیاز برای استفاده از کوبرنیتز در طراحی معماری اینترنت اشیاء می‌پردازد و تکنولوژی‌های مرتبط دیگری را نیز معرفی می‌کند که به هدف‌های مشخص شده در این تحقیق منطبق هستند.
    \begin{figure}[H]
        \center{\includegraphics[width=\textwidth]{figs/ref_figs/iot_monitoring.png}}
        \caption{پایش مبتنی بر کوبرنیتز \cite{Mlynka2022thesis}}
        \label{fig:iot_monitoring}
    \end{figure}
‍}

% \subsection{
%     روش‌های بر پایه توجه
% }
% \label{sec:attn_vqa_approach}
% \paragraph{}{
%     تا قبل از اینکه معماری‌های تبدیل‌کننده معرفی شوند،‌ محققان از مکانیزم توجه
%     بر شبکه‌های عصبی بازگشتی و شبکه‌های کانولوشن استفاده می‌کردند، برای مثال
%     شبکه‌ توجه پشته‌ای
%     \cite{yang2016stacked}
%     ،
%     توجه سلسه‌مراتبی 
%     \cite{NIPS2016_9dcb88e0}
%     و 
%     توجه متقارن
%     \cite{nguyen2018improved}
%     پس از ارائه مکانیزم‌های توجه منتشر شدند.
%     همان‌طوری که از تصویر 
%     \ref{fig:stacked_attention}
%     مشخص است، توجه سلسله‌مراتبی 
%     برای پردازش تصویر از یک شبکه‌ کانولوشن و برای پردازش سوال یا 
%     همان بخش زبانی، از یک 
%     \lr{LSTM}
%     استفاده شده است، سپس با استفاده از مکانیزم پیشنهادی به تناظر میان این دو بخش
%     پرداخته شده است. در هر دو پژوهش مجموعه داده‌های مشهور زبان و تصویر 
%     ارزیابی شده‌ و نسبت به مدل‌های 
%     \lr{SOTA}
%     به خوبی عمل کرده بود. با این‌ حال، این مدل‌ها دارای اشتباهات بسیاری بودند و 
%     نیازمند بهبود بودند تا اینکه مدل‌های تبدیل شونده منتشر شدند و 
%     توانستند به دقت به نسبت بالاتری از این معماری‌ها برسند.
%     \begin{figure}[H]
%         \center{\includegraphics[width=0.7\textwidth]{figs/stacked_attention.png}}
%         \caption{توجه پشته‌ای برای حل مسئله پرسش‌وپاسخ تصویری}
%         \label{fig:stacked_attention}
%     \end{figure}
% }

% \subsection{
%     ادغام روابط اشیاء
% }
% \label{sec:object_relation_approach}
% \paragraph{}{
%     یکی از ایده‌های خلاقانه حل مسئله پرسش‌و‌پاسخ تصویری استفاده از شبکه‌های رابطه‌ای
%     است که در ابتدا روابط بین اشیاء در تصویر و روابط بین کلمات در سوال مشخص
%     می‌شوند و سپس به شبکه‌ای برای پیش‌بینی پاسخ مورد نظر ورودی داده می‌شوند. 
%     از نمونه کارآمد این روش می‌توان به مفاله حل پرسش‌وپاسخ تصویری از طریق 
%     ساختار گرافی 
%     \cite{teney2017graph}
%     اشاره کرد که در ابتدا اشیاء موجود در تصویر و روابط بین آن‌ها،‌
%     و همچنین کلمات موجود در جمله سوال و روابط بین آن‌ها را
%     به صورت گراف توصیف کرده و سپس آن‌ها به شبکه‌ عصبی ورودی می‌دهیم تا 
%     بتواند از طریق آن‌ها به پاسخ مورد نظر برسد.
%     برای بدست آوردن رابطه بین اشیاء در تصویر می‌توان از شبکه‌های رابطه‌ای 
%     و برای بدست آوردن گراف جمله می‌توان از گراف وابستگی گرامری 
%     استفاده کرد. 

%     \begin{figure}[H]
%         \center{\includegraphics[width=0.7\textwidth]{figs/object_relation.png}}
%         \caption{حل پرسش‌وپاسخ تصویری با ساختارهای گرافی}
%         \label{fig:object_relation}
%     \end{figure}
    
% }

% \section{
%     تولید پاسخ برای پرسش‌و‌پاسخ تصویری
% }
% \label{sec:answer_generation}
% \paragraph{}
% {
%     همان‌طوری که در بخش 
%     \ref{sec:approches_vqa}
%     توضیح داده‌شد، تلاش‌های بسیاری برای حل پرسش و پاسخ تصویری از طریق دسته‌بندی
%     شده است، در حالی که حل مسئه از طریق تولید متن توجه چندانی از محققان را 
%     جلب نکرده‌است. راه‌هایی برای حل پرسش‌وپاسخ تصویری با نام‌های
%     \lr{mQA} \cite{gao2015you} 
%     و
%     \lr{AMA} \cite{wu2016ask} 
%     ارائه شد که از شبکه‌های 
%     \lr{LSTM} 
%     برای تولید پاسخ استفاده شده بود. 
%     پس از آن نیز سیستم‌هایی نظیر 
%     \lr{SimVLM} \cite{wang2021simvlm}
%     منتشر شد که هدف آن کاهش هزینه و پیچیدگی آموزش بود. در این 
%     سیستم‌ها حل کردن مسائل مرتبط با زبان و تصویر به هر دو روش
%     تولید پاسخ و دسته‌بندی بررسی شده است. 
%     در نهایت، 
%     \lr{VL-Bart} و \lr{VL-T5} \cite{pmlr-v139-cho21a}
%     یک چهارچوب یکپارچه‌ای را ارائه کردند که توانایی آموزش 
%     کار‌های متفاوتی در کنار یکدیگر را داشت. شکل
%     \ref{fig:vl-bart}
%     کارایی این چهارچوب را در بخش های مختلف نشان می دهد.
%     این مدل‌ها نشان‌ دادند که
%     حل کردن مسائل چند ماژول از طریق تولید متن عملکرد و کارایی بهینه‌تری 
%     از دسته‌بندی دارد. با این حال، از آنجایی که 
%     این مدل‌ها روی مجموعه داده پرسش‌وپاسخ تصویری 
%     \cite{VQA}
%     آموزش داده شده‌اند و پاسخ‌ها در این مجموعه داده به صورت دسته‌بندی داده
%     شده است، همچنان پاسخ دادن به صورت جملات را در بر ندارند.
%     \begin{figure}[H]
%         \center{\includegraphics[width=0.7\textwidth]{figs/VL-BART.png}}
%         \caption{چهارچوب یکپارچه ارائه شده برای حل مسائل چند ماژول به روش تولید متن}
%         \label{fig:vl-bart}
%     \end{figure}

% }

% \section{مجموعه داده‌های منتشر شده در پرسش‌وپاسخ تصویری}
% \label{sec:vqa_datasets}
% \paragraph{}
% {
%     مجموعه داده‌های زیادی در حوزه پرسش‌وپاسخ تصویری ارائه شده است. 
%     با این حال، بسیاری از مجموعه داده‌ها دارای پاسخ‌های تک‌کلمه‌ای هستند 
%     و تعداد کمی از آن‌ها پاسخ‌ها را به صورت جمله نوشته‌اند. 
%     در سال 2019 مقاله 
%     \lr{VCR} \cite{zellers2019recognition}
%     ارائه شد که یک سوال چالشی از تصویری پرسیده شود و در جواب آن باید 
%     پاسخی همراه با دلیل انتخاب پاسخ خروجی داده شود. 
%     این مجموعه داده شامل 
%     290 هزار 
%     پرسش و پاسخ چند گزینه‌ای است که از 110 هزار صحنه فیلم‌ها گرفته شده است. 
%     این مجموعه داده نشان داد که بر خلاف ساده بودن 
%     \lr{VCR}
%     برای انسان‌ها
%     (دقتی بالاتر از 90)
%     ماشین‌ها توانایی چندانی در حل آن ندارند. نویسندگان، مدلی نیز ارائه دادند که این
%     مسئله را حل کند، ولی  طبق ادعای آن‌ها این مسئه همچنان فاصله زیادی تا حل دارد. 
    
%     \begin{figure}[H]
%         \center{\includegraphics[width=1\textwidth]{figs/VCR.png}}
%         \caption{نمونه‌ای مجموعه داده 
%         \lr{VCR}
%         که علاوه بر ارائه پاسخ، نیازمند ارائه دلیلی برای پاسخ است}
%         \label{fig:vcr}
%     \end{figure}
%     مجموعه داده‌ دیگری با نام 
%     \lr{VQA-E} \cite{li2018vqa}
%     منتشر شد که برگرفته در مجموعه داده رسمی پرسش‌وپاسخ تصویری بود. 
%     برای حل این مجموعه داده، علاوه بر ارائه پاسخ 
%     (تک‌کلمه‌ای)
%     باید توضیحی برای پاسخ انتخاب شده ارائه شود. نحوه تولید این مجموعه داده
%     به این صورت است که به ازای هر سه‌تایی پرسش، تصویر و پاسخ یک توضیحی 
%     با توجه به توضیحات تصویر تولید می‌شود. در این مجموعه داده، تلاش بر این بوده
%     که توزیعی مشابه با مجموعه داده رسمی پرسش‌وپاسخ تصویری داشته باشد. 
%     مجموعه داده‌های دیگری نیز به صورت جمعیت‌‌محور ارائه شده‌اند. این مدل مجموعه
%     داده‌ها می‌توان به 
%     \lr{A-OKVQA} \cite{schwenk2022okvqa}
%     اشاره کرد که شامل 25 هزار پرسش است. 
% }

% \section{ مدل‌های از پیش آموزش‌ داده‌شده}
% \label{sec:vlp}
% \paragraph{}
% {
%     پیش‌آموزش مدل‌های زبان-تصویر اخیرا مورد توجه بسیاری واقع شده است. مدل‌های 
%     موجود بیش‌تر بر پایه توجه به خود و معماری تبدیل شونده که در بخش ‌های
%     % \ref{fig:api}
%     و
%     % \ref{sec:transformers}
%     % به صورت جزئی مورد بررسی قرار گرفته‌اند. این مدل‌ها در هر دو بخش 
%     % تصویر و زبان به خوبی عمل کرده‌اند. 
%     % همانطوری که در بخش
%     % \ref{sec:bertlike_arcs}
%     توضیح داده‌شد، این مدل‌ها را می‌توان به دو بخش تک‌جریان و دو جریان دسته‌بندی 
%     کرد. 
%     با توجه به اینکه این مدل‌ها در تولید متن استفاده چندانی نداشته‌اند، در این
%     پژوهش تلاش بر این بوده که مدل‌های از‌ پیش آموزش داده‌شده را برای این هدف 
%     استفاده کنیم. 
% }